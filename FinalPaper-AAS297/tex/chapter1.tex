As a specific group of Asian Americans, Korean Americans have most of the
characteristics of Asian Americans \cite{online:AsianAmericanChar}. They are
respective to the ancestors and elderly, they value the collective interest of a
whole family, and they emphasis the importance of be obedience to parents to
their children. However, one thing that makes Korean American stands out is
their high self-employment rate. According to the book I selected, Korean
American has the highest self-employment
rate \cite[p.~14]{book:ISelected}. 
\par
Self-employment can have a lot of impact on children caring, as stated in the
abstract. First, I want to discuss about the impact on the way parents spend
their time. For self-employed parents, their working time is not fixed. They
might work late to morning one day, and finish work in the early afternoon
another day. As a result of that flexibility and uncertanty, parents generally
will not try to do much of the children caring works. Most of the children
caring work and house work will relay either on grandparents or elder children.
As described in the book I read, elder children need to take care of the younger
ones and do houseworks \cite[p.~22]{book:ISelected}, and as a result, the
children will have less opportunity to communicate with their parents. Another
impact is that the parents will ask their children to help them in their
bussiness in various ways. For example, children
will help their parents tending cash registers, do
cleaning in the restaurant owned by their parents \cite[p.~25]{book:ISelected}.
Generally, those helping will happen after school, during which the children are
more willing to play out with their friends. Helping their parents in grocery or
restaurant may cause some barieers for children to maintain their friendship
with classmates, but is also an opportunity for them to interact and make
friends with adults, who are the clients of their parents' bussiness. The
actually influence will vary from person to person.
\par
In my project, I will focus on how the self-employment of parents influences
their children. In particular, the influence on the relationship between
children and their classmates, between children and their parents, and between
siblings. I will collect data by interviewing a second generation Korean
American and her family who lives in Chicago whose parents owned a small
grocery store when she was young.
The detail of the interview will be elaborated in section 3.
