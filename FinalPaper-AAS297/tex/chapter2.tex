I chose the book \emph{Caring Across Generations: The Linked Lives of Korean
American Families} as the reading of my project. As its title says, it about
caring between family members, but mainly focused on the caring between
generations. The population of that book is the Korean immigrants to American,
mostly consists of one to third generation of immigrants. Since the first
generation immigrants Korean parents are generally not good at English, they
relay their children who usually have better English skills as the interface to
interact with outside world. Further more, the education of the parents got from
South Korea is not usually admitted by the companies in the U.S., so they are
facing downward mobility, which will add financial and spiritual pressure to the
family. When the children are old enough, they will work in some part-time jobs,
or help in their parents' business in order to give their family some support.
As parents getting older, and children growing up, children have even more
obligation to take care of their parents, according to the cultural tradition.
\par
So, in that book, the authors are focusing on how and by which extend the
difficulties those immigrant parents facing can change the direction of
caring, and how that changing can influence their relationship with their
children. Since Korean American has a significantly high rate of
self-employment, there is no way for the authors of the book to omit the effect
of self-employment.
\par
The biggest help of that book for my project is giving me a way as a
semi-outsider(The reason is that I am not Korean American, but I am Asian
American) to peel into the Korean culture and life and make linkages to my own
experience, and eventually aid my understanding of Korean American. Another
usage of that book is to prepare my interview questions. The book gave me a lot
of ideas of what my interview questions will be, and provided me some background
knowledge. For example, the yes branch of the second question, the third
question, the fifth question, the eighth and ninth questions are all created
after reading the cases of the sample Korean Americans of that book. And my
knowledge of the term ``anjong'' for the last two questions comes directly from
that book \cite[p.~40, p.~41]{book:ISelected}, which means ``establishment,
stability, or security,''
