The interview is the main part of my project. And I want to show it in the
following order: 1. About how I found the interviwee family and my data
collection stratergy 2. The demographic structure of that family, and what their
home looks like. 3. Their response to my interview questions.
\subsection{How I found the family; how I collected data}
The interviewee family is the host of my Airbnb house of my Chicago trip during
the Thanks Giving holiday. When I arrived their home, it was afternoon, and was
strongly snowing out side. I had nothing to do, so I chated with them, and found
they came from South Korean, and the hostess Ann (psudoname), whom I contacted
directly on Airbnb, was born in America, and had just graduated from 
University of Chicago last year. She is a singer and song writer now.
\par
I decided to use interview to collect data. During the interview, I decided not
to take notes nor to make record. So, the answers of the interview question are
recovered from my memory. 
\subsection{Brief information about that family}
The family lives in a house in the suburban area of Chicago. Honestly, when I
first stepped into their home, I couldn't see difference from the ordinary
American homes, except a ink picture of mountain and river with Korean
Characters on it. Everything else is almost the same as other American
families. However, their home is not as clean as other American families. 
\par
There are four people in her family:her parents, her elder brother who is
working in Chicago, and herself. Also, her boy friend, who is a white boy, will
come to visit her occationally. Her parents immigranted to American thirty years
ago, and still have difficulty to speak good English. Ann told
me, when she was young, her parents had a small restaurant in the nearby
neighborhood. They're really friendly and gave me a lot help when I was in
Chicago.
\subsection{Their response to my interview question}
Now is their response to my interview questions.
\begin{itemize}
  \item First question I asked Ann, and her answer is yes.
  \item Second question, Ann told me, sometimes, she was really reluctant to
  work in her parents' restaurant because she had to do ordering for the
  customs and needed to face a lot of unknown people, which would make her feel
  really shy. However, she also told me that she is now grateful for that 
  experience because that made her open and confident. When she was not at her
  parents' resaturant, her brother would go over and do her jobs for their
  parents.
  \item Third question, Ann told me her parents seldom told her about how hard
  they made to immigrant, but they always stress their expectation for her, to
  get grade, to get into good university. Ann said, that at the time she
  received the offer from University of Chicago, her parents were so excited
  that they cried out. Later on, her parents told every one they knew about that
  good news, and said she is their prond. Ann said, her parents' expectation
  really gave her a lot of stress, but it was that pressure that made her to
  such achievement.
  \item Fourth question: She said she had done that before, like refuse to do
  the homework. But later she found that's useless, and would cause more
  trouble, so she stoped doing that.
  \item The fifth question she had already answered in the first one, so I
  omitted it.
  \item This question, Ann said they didn't speak out directly, but she knew
  they love her. Her parents were usually very busy, and she spent most of her
  time with her brother.
  \item Her answer to sixth question is simple, just ``of course''.
  \item Ann's answer is yes. And she told me, she was admitted to other pretty
  good universities on the east coast, but she chose UChicago because it is
  close to home, and she can help her parents when they miss her.
  \item Ann haven't seen her parents stressed out, but later her parents told me
  that they felt really helpless several times, because of finiantial issues and
  others.
  \item I skipted the this question, since I tought would be a little bit rude
  to ask.
  \item They heard of that term.
  \item I asked the question to Ann and her parents seperately. Ann's answer is
  her parents are healthy. Her parents' answer is Ann and her brother
  are happy and have a good life.
\end{itemize}